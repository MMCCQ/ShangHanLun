
% preamble.tex
% Encoding: UTF-8

%“em”是相對长度單位,相当於所用字体中大寫“M”的寬度。
%“ex”是相對长度單位,相当於所用字体中小寫“x”的高度,此高度通常为字体尺寸的一半。

\documentclass[b5paper,twoside,zihao=-4,notitlepage,openany,UTF8]{ctexbook}

\usepackage{syntonly}
%	\syntaxonly%編譯時只檢查語法,不輸出PDF。

\usepackage{ctex}
	\ctexset{
		section = {
			format += \zihao{-4} \heiti \raggedright,
			name = {,.},
			number = \arabic{section},
			beforeskip = 1.0ex plus 0.2ex minus .2ex,
			afterskip = 1.0ex plus 0.2ex minus .2ex,
			aftername = \hspace{0.1em}
		},
		chapter = {
			format += \zihao{-3} \heiti,
			name = {,.},
			number = \chinese{chapter},
			beforeskip = 1.0ex plus 0.2ex minus .2ex,
			afterskip = 1.0ex plus 0.2ex minus .2ex,
			aftername = \hspace{0.1em}
		}
	}

\usepackage{xeCJK}
	\xeCJKsetup{
		PunctStyle = kaiming
	}
	\setCJKmainfont[Path=ttf/]{TH-Sung-TP0}
	\setCJKsansfont[Path=ttf/]{TH-Sung-TP0}
	\setCJKmonofont[Path=ttf/]{TH-Sung-TP0}

%	\setCJKmainfont{SimSun}%设置正文罗马族的 CJK 字体
%	\setCJKsansfont{SimSun}%设置正文无衬线族的 CJK 字体
%	\setCJKmonofont{SimSun}%设置正文等宽族的 CJK 字体

\usepackage{fontspec}
%	\setmainfont{SimSun}
%	\setmainfont[Path=ttf/]{TH-Sy-P0}
%	\setsansfont{SimSun}

\usepackage{geometry}
	\geometry{left=2.5cm,right=2.5cm,top=2cm,bottom=2cm}

\usepackage[stable]{footmisc}

\usepackage{graphicx}
	\graphicspath{{pics/}}

\usepackage{xcolor}

\usepackage{tocloft}
	\cftbeforesecskip 0.5ex
	\cftbeforechapskip 0.5ex
	\cftbeforepartskip 1ex
	\cftsecindent 4em
	\cftchapindent 2em
	\cftpartindent 0em

\usepackage{makeidx}
	\makeindex

\usepackage{tcolorbox}
	\tcbuselibrary{skins, vignette, breakable, theorems, fitting}

%\nofiles
\pagestyle{headings}
%\punctstyle{kaiming}
\CJKsetecglue{}%完全禁用汉字与其他内容间的空格
%\raggedright
\setlength{\parindent}{0em}%段落中第一行缩进量
\setlength{\parskip}{3ex}%段落间距
\setcounter{secnumdepth}{4}
\setcounter{tocdepth}{4}

\title{傷寒雜病論
	\footnote{
		唐弘宇按:本論的原名現已不可考,可能本來就沒有名字。至於「张仲景方」、「张仲景辨傷寒」、「仲景全書」、「傷寒卒病論」等名,与「傷寒雜病論」一樣,皆是後人杜撰,沒有優劣之分。我選取「傷寒雜病論」作为書名的原因有二,一是這個名字已被大乑所熟知,二是這個書名概括了本論的兩大部分,即傷寒部分和雜病部分。有很多老師和同學給我提意見,説這個名字不好,應改成XXX。我認为名字叫什麼不必太過追究,陸淵雷説的好:「其菁華,在於諸方之症候用法」。
	}
}

\author{
	张仲景{\hfill}述\\
	王叔和撰次\\
	{\includegraphics[width=3em]{tanghongyu.jpg}}校訂
\and
	{\includegraphics[width=3em]{maomin.jpg}}\\
	{\includegraphics[width=3em]{yejianqiao.jpg}}\\
}

\date{\today}

\endinput